\textbf{\large Результаты}

Замеры времени работы программы с различными конфигурациями (время указано в микросекундах).

\begin{tabular}{|c|c|c|c|c|}\hline
\diaghead{\theadfont Diag ColumnmnHead II}%
{Раз-\\мер сетки}{Число про-\\цессов}&
\thead{1}&\thead{2}&\thead{4}&\thead{8}\\
\hline
1000 & 6 327 & 22 351 & 7 021 & 7 401\\
\hline
10 000 & 157 941 & 102 521 & 97 145 & 121 873\\
\hline
30 000 & 1 123 839 & 741 983 & 452 901 & 669 230\\
\hline
60 000 & 3 522 883 & 1 775 134 & 1 052 952 & 1 525 567\\
\hline
100 000 & 8 453 963 & 4 464 091 & 2 963 146 & 3 153 957\\
\hline
\end{tabular}

По таблице видно, что лучшая производительность достигается при использовании 4 процессов.

