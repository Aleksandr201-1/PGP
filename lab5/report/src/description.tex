\textbf{\large Описание программы}

ОСновные функции:
\begin{itemize}
    \setlength\itemsep{0.0em}
    \item $histogram$, 
    \item $addKernel$,
    \item $scanKernel$,
    \item $CountingSortWrite$
\end{itemize}

При помощи функции $histogram$ происходит подсчёт одинаковых элементов с использованием атомарной операции $atomicAdd$.

Функция $addKernel$ нужна для добавления массива сумм к результирующему масииву после работы $scanKernel$.

Функция $scanKernel$ рекурсивно считывает блоки данных по 1024 элемента и строит префиксную сумму.

Функция $CountingSortWrite$ записывает отсортированный массив на основе префиксной суммы.

Алгоритм сортировки состоит в последовательном применении гистограммы, сканирования и записи.

Код функций:

\begin{lstlisting}[basicstyle=\normalfont, language=C++]
__global__ void histogram (uint32_t *count, DATA_TYPE *data, uint32_t n) {
    uint32_t idx = blockIdx.x * blockDim.x + threadIdx.x;
    uint32_t offsetX = gridDim.x * blockDim.x;

    for (uint32_t i = idx; i < n; i += offsetX) {
        atomicAdd(count + data[i], 1);
    }
}

__global__ void addKernel(uint32_t* data, uint32_t* sums, uint32_t sums_size) {
    uint32_t blockId = blockIdx.x + 1;
    while (blockId < sums_size) {
        uint32_t idx = blockId * blockDim.x + threadIdx.x;
        if (blockId != 0) {
            data[idx] += sums[blockId];
        }
        blockId += gridDim.x;
    }
}

__global__ void scanKernel(uint32_t* data, uint32_t* sums, uint32_t sums_size) {
    uint32_t blockId = blockIdx.x;
    //uint32_t idx = blockId * blockDim.x + threadIdx.x;
    while (blockId < sums_size) {
        __shared__ uint32_t tmp[1024];
        uint32_t step = 1, a, b, tmp_el;
        tmp[index(threadIdx.x)] = data[blockId * blockDim.x + threadIdx.x];
        __syncthreads();
        while (step < BLOCK_SIZE) {
            if ((threadIdx.x + 1) % (step << 1) == 0) {
                a = index(threadIdx.x);
                b = index(threadIdx.x - step);
                tmp[a] += tmp[b];
            }
            step <<= 1;
            __syncthreads();
        }
        if (threadIdx.x == 0) {
            sums[blockId] = tmp[index(BLOCK_SIZE - 1)];
            tmp[index(BLOCK_SIZE - 1)] = 0;
        }
        __syncthreads();
        step = 1 << 10 - 1;
        while (step >= 1) {
            if ((threadIdx.x + 1) % (step << 1) == 0) {
                a = index(threadIdx.x);
                b = index(threadIdx.x - step);
                tmp_el = tmp[a];
                tmp[a] += tmp[b];
                tmp[b] = tmp_el;
            }
            step >>= 1;
            __syncthreads();
        }
        data[blockId * blockDim.x + threadIdx.x] = tmp[index(threadIdx.x)];
        blockId += gridDim.x;
    }
}

__global__ void CountingSortWrite (DATA_TYPE *data, uint32_t *count, uint32_t size) {
    uint32_t idx = blockIdx.x * blockDim.x + threadIdx.x;
    uint32_t offsetX = gridDim.x * blockDim.x;
    uint32_t idy = blockIdx.y * blockDim.y + threadIdx.y;
    uint32_t offsetY = gridDim.y * blockDim.y;

    for (uint32_t i = idx; i < DATA_MAX_EL; i += offsetX) {
        uint32_t k = count[i];
        for (uint32_t j = idy + k; j < count[i + 1]; j += offsetY) {
            data[j] = i;
        }
    }
    for (uint32_t i = idx + count[DATA_MAX_EL]; i < size; i += offsetX) {
        data[i] = DATA_MAX_EL;
    }
}
\end{lstlisting}

\vspace{15pt}

