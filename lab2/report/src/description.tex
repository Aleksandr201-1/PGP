\textbf{\large Описание программы}

Для преобразования изображения алгоритмом SSAA необходимо разбить исходное изображение на $n \times m$ блоков, где $n$ ~--- соотношение старой ширины к новой, $m$ ~--- старой высоты к новой. Внутри каждого блока происходит суммирование всех значений и деление результата на площадь блока. Таким образом находится средний цвет. Для хранения данных изображения используется текстурная память.

Вызов $SSAAkernel$ происходит с количеством нитей на блок ~--- 1024 и количеством блоков ~--- 1024.
В самом $SSAAkernel$ я вычисляю общий индекс исполняемой нити $idx$ который и будет индексом в массиве при условии $idx < block\_count$, где $block\_count$ ~--- количество блоков, на которое поделено исходное изображение, смещение $offset$, которое позволяет найти следующий обрабатываемый блок для нити.

Код функции:

\begin{lstlisting}[basicstyle=\normalfont, language=C++]
__global__ void SSAAkernel (uint32_t old_w, uint32_t old_h, uint32_t new_w, uint32_t new_h, uint32_t *new_buff) {
    uint32_t idx = blockIdx.x * blockDim.x + threadIdx.x;
    uint32_t offset = blockDim.x * gridDim.x;

    uint32_t block_w = old_w / new_w, block_h = old_h / new_h;
    uint32_t block_size = block_w * block_h, block_count = new_h * new_w;
    while (idx < block_count) {
        uint32_t i0 = idx / new_w, j0 = idx % new_w;
        uint32_t r = 0, g = 0, b = 0, a = 0;
        for (uint32_t i1 = 0; i1 < block_h; ++i1) {
            for (uint32_t j1 = 0; j1 < block_w; ++j1) {
                uint32_t color = tex2D(img_tex, j1 + j0 * block_w, i1 + i0 * block_h);
                r += (color >> 24) & 255;
                g += (color >> 16) & 255;
                b += (color >> 8)  & 255;
                a += color  & 255;
            }
        }
        r = (r / block_size) << 24;
        g = (g / block_size) << 16;
        b = (b / block_size) << 8;
        a = a / block_size;
        new_buff[i0 * new_w + j0] = r + g + b + a;
        idx += offset;
    }
}
\end{lstlisting}

\vspace{15pt}

