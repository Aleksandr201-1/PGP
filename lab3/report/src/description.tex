\textbf{\large Описание программы}

Для преобразования изображения алгоритмом Махаланобиса, нужно предварительно вычислить вектора средних значений пикселей и ковариационные матрицы. Их вычисление происходит на CPU и копируется в разделяемую память.

Далее при помощи формулы Малаханобиса находится класс с наименьшим номером, к которому был отнесен соответствующий пиксель.

Вызов $MahalanobisKernel$ происходит с количеством нитей на блок ~--- 1024 и количеством блоков ~--- 1024.
В самом $MahalanobisKernel$ я вычисляю общий индекс исполняемой нити $idx$ который и будет индексом пикселя, смещение $offset$, которое позволяет найти следующий обрабатываемый блок для нити.

Код функции:

\begin{lstlisting}[basicstyle=\normalfont, language=C++]
__global__ void MahalanobisKernel (uint32_t w, uint32_t h, uint64_t nc, uint32_t *data) {
    int idx = blockIdx.x * blockDim.x + threadIdx.x;
    int offsetX = gridDim.x * blockDim.x;

    for (uint64_t i = idx; i < w * h; i += offsetX) {
        uint32_t color = data[i];
        double max = getArg(color, 0);
        uint32_t curr_class = 0;
        for (uint64_t k = 1; k < nc; ++k) {
            double tmp = getArg(color, k);
            if (max < tmp) {
                max = tmp;
                curr_class = k;
            }
        }
        data[i] += curr_class;
    }
}
\end{lstlisting}

\vspace{15pt}

